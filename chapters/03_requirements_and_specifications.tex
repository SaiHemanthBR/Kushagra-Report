\documentclass[../Report.tex]{subfiles}

\begin{document}

\chapter{Requirements and  Specifications}

\section{Software Requirements}

\subsection{Functional Requirements}

\begin{enumerate}
  \item User should be able to login using Email/Google account.
  
  \item User should be able to take/use a picture to get predictions about the crop/crop disease.
  
  \item Save the predictions in an database and access it from other devices.
  
  \item Bookmark predictions.
  
  \item Generate Report using the predictions.
  
  \item Save report as PDF and share it easily using email or chat application (like iMessages, WhatsApp).

\end{enumerate}

\subsection{Non-Functional Requirements}

\begin{description}
  \item[Performance:] Performance is essential for user experience. Performance is critical in both mobile app and backend. Slow performance 
  is detrimental is app usage. With the help of optimization techniques, the entire processing pipeline is tested using strictly timed 
  unit tests.
  
  \item[Scalabilty:] This application is built to be have thousands of concurrent users. So, this project was developed with scalability in 
  mind. The server is built on flask which is light-weight, requiring less resources. The web server is containerized, so, depending on the 
  usage, docker can automatically start or stop instances.

  \item[Usability:] In mobile app, localization support allow users to use the app in their preferred local language. Scalable, optimized 
  backend means low latency for information processing, less downtime and better usability.

  \item[Availability:] This app is available for both iOS and Android. Since android platform is omnipresent in mobile market, user 
  penetration can easily be archived. The small share of iOS market is also not left alone. In, later versions, image classification
  can be performed on-device, making internet not a necessity.

  \item[Security:] All requests to server are secured using TLS. User data is saved on device using on-device encryption. Data on 
  firebase is automatically using user password.

  \item[Cost and Maintainability: ] Firebase is a pay-as-you-go model, lowering the costs considering that maintenance costs are 
  essentially non-existential.
\end{description} 

\section{System Specifications}

\subsection{Software Specifications}

\begin{description}
  \item[Pytorch:] PyTorch is an open-source machine learning library based on Torch library. PyTorch is a Tensor computing library and also 
  a deep learning library with automatic differentiation system. In can be used to build state-of-the-art Neural Networks which are used in
  self-driving cars.\par
  The crop detection and crop disease detection is done using a Convolutional Neural Network(CNN) built using ResNet50 as a feature extractor.
  The current models have an accuracy of 92\% and 90\%.

  \item[NumPy:] NumPy is a numerical library that is used for accelerated tensor operations. This help is speeding up the computation by 
  taking advantage of SIMD instructions on modern CPUs and GPUs. This library is useful for preprocessing data before classification.

  \item[Matplotlib:] Matplotlib is a plotting library for python. This is useful for EDA and performing error analysis while training 
  and testing the AI models.

  \item[Pillow:] Pilliow is a Python Image Library is an free and open-source that supports opening, modifying, saving images. This is used to convert data
  sent over internet as bytes to images that can then be used in the AI models for inference. PIL is also used to load images from disk for
  training the models.

  \item[Colab:] Google Colab is a free cloud service which helps in developing deep learning applications using popular libraries such 
  as Keras, TensorFlow, PyTorch. In runs on Google servers with GPU or TPU acceleration which helps in reducing the training period for 
  the AI model. Giving team, the ability to iterate and improve quickly.

  \item[Flask and Docker:] Flask is a light-weight web framework for python. This can be used to run the inference server. Flask is ran
  inside a docker container. Docker is a set of platform as a service products that uses OS-level virtualization to deliver software in 
  packages called containers. These container can easily be scaled up or down to automatically adjust for usage and network traffic.

  \item[Android Studio:] Android Studio is the official IDE for developing native Android Apps. Android Studio provides a unified  
  environment where you can build apps for Android phones, tablets, Android Wear, Android TV, and Android Auto. Structured code modules 
  allow you to  divide your project into units of functionality that you can independently build, test, and debug.\par
  Android's Location library can be used to obtain user's location. Google Maps can we used to display user's location and provide 
  additional information.

  \item[Xcode:] Xcode is the official IDE for developing native iOS apps. Xcode provides a unified environment for developing native apps 
  for all apple devices and services. Swift programming language is used to write apps in Xcode. CoreLocation framework can be used to 
  obtain user's location. MapKit (Apple Maps) can be used to display user's location and provide additional information.\par

  \item Apps can be localized to make them be available in multiple language, so more people can use the app without english being a 
  hindrance.

  \item[Firebase:] Firebase is a BaaS (Backend as a Service) provided by Google. Firebase is scalable, distributed and secure by design. It is especially
  geared towards business apps, with the intention of helping businesses grow their user bases and increase their profits through their mobile apps.\par
  Firebase provides authentication(Auth), Database (Firestore), Cloud Storage (Storage), Analytics as a service.
\end{description}

\subsection{Hardware Specifications}

\begin{description}
  \item[User Device Requirements:]
  \item Any Android phone running Android Oreo or above.
  \item iPhone or iPad running iOS 13 or above.
  \item An internet connection.
  \item 
  
  \item [Minimum Server Requirements:]
  \item Processor: Any x86 CPU with clock speed of 2.5GHz and above.
  \item RAM: 2GB.
  \item Storage: At-least 10 G.B.
\end{description}
Above mentioned server requirements are minimum and need to be automatically scaled depending on usage and traffic.

\end{document}