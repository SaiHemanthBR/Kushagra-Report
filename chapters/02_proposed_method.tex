\documentclass[../Report.tex]{subfiles}

\begin{document}

\chapter{Proposed Model}

\section{Introduction to the characteristics of the problem}

Though still in the beginning of its journey, \acrshort{ml}-driven farms are already evolving into artificial intelligence systems.
At present, machine learning solutions tackle individual problems, but with further integration of automated data recording,
data analysis, machine learning, and decision-making into an interconnected system, farming practices would change into with
the so-called knowledge-based agriculture that would be able to increase production levels and products quality.\par

\noindent
The main characteristics include:-
\begin{description}
  \item[Crop Detection:] This feature allows the automated machines in future to differentiate the crops and suggest appropriate 
  techniques in improving crop quality.
  
  \item[Disease Detection:] The current existing method for plant disease detection is simply naked eye observation which sometimes may not 
  give accurate results.With the help of ML the past disease remedies can be applied  and get immediate solutions to prevent further catastrophe
  in farming.
\end{description}

\section{Design Challenges}

\begin{description}
    \item[Creating an Accurate AI Model:] The primary design challenge is to creating an AI model with acceptable accuracy. CNN\cite{cnn}
    can be used for image detection. A simple NN can be created using transfer learning. This base model is essential to design an 
    Minimum Viable Product (MVP). This MVP is essential for data collection for improving the AI model with Data Engine.
    
    \item[Managing Design Influence:] Software is know for its high plasticity and the software requirements changes as stakeholders for
    the app change. This need to be primary consideration for design of any API/Data Models that are crucial for app's performance.

    \item[Ethical Practices:] Deep learning model are inherently data hungry and these also tend to perform better with more data. So
    data collection is essential but this also presents an ethical question about privacy. So, any data collection should only be to 
    improve the AI models and not breach user privacy. Steps need to be take to educate user's what kind of data is collected.

    \item[Performance:] Any app need to be fast and scalable. All computationally intensive processes that need to be 
    done of user's device need to be run in the background to prevent app from becoming unresponsive. This can be archived using 
    threads. A thread of computation can be created for any heavy computational or network activity. This asynchronous nature can 
    cause race conditions. This race conditions can be dealt with thread safe data models and semaphores.
    
    \item[Scalability:] Backend also need to scale based on usage, traffic. So care need to be taken to design API/Date models that 
    can easily scale in realtime based on requirements. Containerizing the backend makes it easy for scalability. Docker is an software 
    environment that can be used to create, maintain and run the containers.
\end{description}

\section{Proposed Solution}

Considering the limitations of existing systems, it seems obvious that we need to have a better system. We have developed a simple 
mobile application and helps farmers with learning about their crops, diagnose any crop diseases without having to waste their resources
on futile solutions.\par
Automatic Plant/Disease Identification is the most promising solution that can be used to partially solve this problem. Solving the entire
problem requires an update of basic infrastructure and education. Saying that, this is a small starting step to that future.\par
Improving the performance of the Mobile-based plant identification helps users identify the crop and obtain the information about the 
crop/crop disease.\par

\end{document}