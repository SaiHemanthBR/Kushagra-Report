\documentclass[../Report.tex]{subfiles}

\begin{document}

\chapter{Introduction}

\section{Motivation}
Intelligence has been considered as the major challenge in promoting economic potential and production efficiency of precision
agriculture. Modern agriculture seeks to manage crops in controlled environments that are able to improve the production of plants or
duplicate the environmental conditions of specific geographical areas to obtain imported products locally. \par
Now it is possible to obtain highly accurate status of crops and form reasonable decisions to manage irrigation, enrich the soil
nutrition in agricultural scenes. Automatic Plant Image Identification is the most promising solution towards bridging the botanical
taxonomic gap, which receives considerable attention. \par
As the technology advances, sophisticated models have been proposed for automatic plant identification. With the popularity of technology, 
by the use of smartphones, many plant photos have been acquired. This data can be analysed and used to identify a plant. Improving the 
performance of the Mobile-based plant identification helps users identify the crop and obtain the information and geolocation of the 
respective crop.\par
Government agencies and agricultural managers require information on the spatial distribution and area of cultivated crops for planning 
purposes. Agencies can more adequately plan the import and export of crop there by reducing price changes and stress on consumers.

\section{Problem Statement}
Develop a mobile application that can identify crop using the field photo of the crop. The application allows the user to take photos 
and automatically detects the crop. The photo of the crop along with its information and geolocation, are stored in a database. To ensure 
farmers regarding latest methodologies and techniques in the process of cultivation and suggest the necessary remedies for the 
diseases in crops.\par


\section{Project Report Organization}


\end{document}