\documentclass[../Report.tex]{subfiles}

\begin{document}

\chapter{Introduction}

\section{Motivation}

Intelligence has been considered as the major challenge in promoting economic potential and production efficiency of precision
agriculture. Modern agriculture seeks to manage crops in controlled environments that are able to improve the production of plants or
duplicate the environmental conditions of specific geographical areas to obtain imported products locally.\par

Now, it is possible to obtain highly accurate status of crops and form reasonable decisions to manage irrigation, enrich the soil
nutrition in agricultural scenes. Automatic Plant Image Identification is the most promising solution towards bridging the botanical
taxonomic gap, which receives considerable attention.\par

As the technology advances, sophisticated models have been proposed for automatic plant identification. With the popularity of technology, 
by the use of smartphones, many plant photos have been acquired. This data can be analysed and used to identify a plant. Improving the 
performance of the mobile-based plant identification helps users identify the crop and obtain the information and geolocation of the 
respective crop.\par

Government agencies and agricultural managers require information on the spatial distribution and area of cultivated crops for planning 
purposes. Agencies can more adequately plan the import and export of crop there by reducing price changes and stress on consumers.

\section{Problem Statement}

Develop a mobile application that can identify crop using the field photo of the crop. The application allows the user to take photos 
and automatically detects the crop. The photo of the crop along with its information and geolocation, are stored in a database. To ensure 
farmers regarding latest methodologies and techniques in the process of cultivation and suggest the necessary remedies for the 
diseases in crops.\par


\section{Project Report Organization}

In this \textit{``Project Report''} a detailed description about the design challenges, proposed methodologies and the implementation of 
application to solve the real world problem is given. Different functionalities of the application are broken down into modules and 
explained with the help of use case diagrams and class diagrams. The working model of the application is shown using screenshots in this 
report.\par

\noindent
This report is organized into six chapters. After this introductory chapter,
\begin{description}
    \item[Chapter 2:] Describes the characteristics of the problem, design challenge and proposed solution
    
    \item[Chapter 3:] Provides software requirements which includes functional requirements, non-functional requirements, 
    high level architecture of the proposed system and system specifications which includes software requirements and hardware requirements.
    
    \item[Chapter 4:] Can be elaborated either in top down number or bottom up manner based on the development strategy adapted. 
    It describes use-case diagrams, class diagrams, activity diagrams, architecture diagrams.
    
    \item[Chapter 5:] In this chapter, implementation and testing is discussed in detail.
    
    \item[Chapter 6:] Describes the conclusion and future enhancements.
\end{description}
\end{document}